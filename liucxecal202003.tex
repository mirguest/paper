% Created 2020-04-09 周四 15:45
% Intended LaTeX compiler: pdflatex
\documentclass[11pt]{article}
\usepackage[utf8]{inputenc}
\usepackage[T1]{fontenc}
\usepackage{graphicx}
\usepackage{grffile}
\usepackage{longtable}
\usepackage{wrapfig}
\usepackage{rotating}
\usepackage[normalem]{ulem}
\usepackage{amsmath}
\usepackage{textcomp}
\usepackage{amssymb}
\usepackage{capt-of}
\usepackage{hyperref}
\usepackage[UTF8]{ctex}
\author{林韬}
\date{\today}
\title{刘春秀ECAL优化报告及代码阅读笔记}
\hypersetup{
 pdfauthor={林韬},
 pdftitle={刘春秀ECAL优化报告及代码阅读笔记},
 pdfkeywords={},
 pdfsubject={},
 pdfcreator={Emacs 26.3 (Org mode 9.3.6)}, 
 pdflang={English}}
\begin{document}

\maketitle
\tableofcontents


\section{报告主要内容}
\label{sec:org820cb76}

\begin{itemize}
\item 第2页,主要内容包括BGO晶体的模拟和重建,性能研究
\item 第3页,研究动机
\begin{itemize}
\item 优化晶体的颗粒度
\begin{itemize}
\item 粒子鉴别的性能,特别是高能 \(\pi^0\) 衰变产生的 2\(\gamma\)
\item 区分 \(\gamma\) 和 K\textsubscript{L},基于cluster的时间等分布
\item 重建能量泄露的可能性
\end{itemize}
\item crystal的排布
\begin{itemize}
\item 单元的大小
\item 层数
\item 每层的深度
\end{itemize}
\item CEPC中ECAL的几何
\begin{itemize}
\item 晶体的大小,wrapper(?)
\item 估计所有晶体的体积
\item 估计电子学通道数
\item 造价估计
\end{itemize}
\end{itemize}
\item 第4页,基于GEANT4 10.5的探测器构建
\begin{itemize}
\item 一个简化版本的晶体量能器模块
\item Matrix module
\begin{itemize}
\item 60x60x60 cells
\item 读出单元 10mm x 10mm x 10mm
\item 探测器阵列前端到探测器中心的距离为 1835mm
\end{itemize}
\item GEANT4仅仅模拟了cell中的沉积能量
\item 10mm 的单元对于桶部 90度 部分的立体角为 0.31度
\item{$\square$} ?如何计算的
\item 在 r-phi 平面,大约能放 19 个 Matrix module
\item 在 z 方向,桶部量能器的一半为 2245mm,所以大概放 7 个 Matrix module
\end{itemize}
\end{itemize}

\begin{verbatim}
>>> import math
>>> math.asin(300./1835.)
0.16422492996270302
>>> math.asin(300./1835.)*2
0.32844985992540604
>>> 2*math.pi / (math.asin(300./1835.)*2)
19.129815761244515
>>> 1150./60
19.166666666666668

>>> 2*2245. / 600
7.483333333333333

\end{verbatim}

\begin{itemize}
\item 第5页,产生的模拟样本,为single particle, \(\gamma\) 和 \(\pi^0\)
\begin{itemize}
\item \(\gamma\) 三个能量点 98 GeV,100 GeV,102 GeV
\item \(\pi^0\) 一个能量点 30 GeV
\item 动量方向,从探测器中心飞向晶体中心。
\item{$\square$} 但是报告中的 30 是怎么回事?
\end{itemize}

\item 第6页,每层cluster重建
\begin{itemize}
\item 流程:clustering,finding seed,finding shower
\end{itemize}

\item 第7页,研究了相关的性能
\begin{itemize}
\item 包括高能 \(\gamma\) 能量泄露时的修正、两个 \(\gamma\) 和 \(\pi^0\) 衰变的区分
\item ECAL 探测器的配置
\begin{itemize}
\item 将3层合并,也就是每层变成3cm
\item 在每层进行cluster重建
\item 考虑探测器由不同的厚度,30cm/27cm/24cm,分别对应10层/9层/8层
\end{itemize}
\end{itemize}

\item 第8页,展示了一个例子,100 GeV 的 \(\gamma\) 例子在每层的分布
\begin{itemize}
\item 报告中提到,如果没有任何的能量阈值限制,将会重建出很多的cluster
\item 分布给了cell大小为 1cm x 1cm 和 2cm x 2cm
\end{itemize}

\item 第9页,给出每层的cluster能量和xy平面R的关系
\item 第10页,如果假设hit、seed、cluster energy的阈值,小能量的cluster就会消失
\begin{itemize}
\item 在后面的研究中,使用了能量重建时能量最大的cluster
\end{itemize}
\item 第11页,给出了能量最大的cluster对应的能量和总能量的比值分布

\item 从12页开始,研究纵向的能量泄露修正
\begin{itemize}
\item{$\square$} 但我对于 longitudinal 没有太多的概念。和地球是一样的,连接南北极的线
\item 应该就是径向
\end{itemize}
\item 第13页,产生的样本:三个能量点的 \(\gamma\) 。研究时主要看能量的泄露。
\item 第14页,沉积能量和每层的关系。此处每层厚度3cm
\begin{itemize}
\item 两张图分别是正常的y坐标和对数的y坐标。应该是Profile图,y轴应该是沉积能量的平均值。
\item 但为什么分布是这样的呢?应该和shower有关系。
\begin{itemize}
\item 猜测:刚开始时发生shower,开始簇射产生次级粒子,这些次级粒子也往横向、径向发展。
\item 当到达一定的径向位置后,shower的横向位置变大,所以沉积了更多的能量。
\item 最后逐渐减少。
\item{$\square$} 这部分应该之前学习过的。需要确认。
\end{itemize}
\end{itemize}
\item 第15页给出修正的方法
\begin{itemize}
\item 利用前10层或9层的沉积能量,拟和出随簇射深度的变化。
\item 然后将该函数在30cm或27cm至60cm的积分值作为修正量。
\item{$\square$} 其实我这里还没有看到到底泄露了多少能量
\item 修正函数 \(\frac{dE}{dt} = E_0 b \frac{(bt)^{a-1}e^{-bt}}{\Gamma(a)}\)
\item 三个参数为 \(E_0\) \(a\) \(b\)
\end{itemize}
\item 第16页给出了修正前后的差值,以 102 GeV gamma 为例
\item 第17页给出三个能量点的重建能量分布。
\begin{itemize}
\item 三条曲线
\item 黑色表示仅用30cm厚度得出的能量?是重建能量?还是原初的能量?看图中的介绍,好像是沉积能量。
\item 蓝色表示加上额外的修正值后的分布
\item 红线表示?
\end{itemize}
\end{itemize}

\section{代码相关内容}
\label{sec:org183175e}

\begin{itemize}
\item 默认的探测器参数
\begin{itemize}
\item 每个 tile 为 \(10\textrm{mm} \times 10\textrm{mm} \times 10\textrm{mm}\)
\item 每个 layer 由 \(60\times60\) 的 tile 构成
\item 最终构建的 Ecal 为 60 层 layer
\end{itemize}
\end{itemize}
\end{document}
